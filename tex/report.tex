\documentclass[a4paper,12pt]{article}


\usepackage{amssymb,amsmath,array}
\usepackage{hyperref}
\usepackage{bm}

\usepackage[a4paper, left=3cm, right=2.5cm, top=3cm, bottom=3cm]{geometry}
% language and encoding
\usepackage[utf8]{inputenc}
\usepackage[T1]{fontenc}
\usepackage[english]{babel}
% this package load language specific quotes and enables you to write
% \enquote{text} instead of "`text"' or something like that.
% \usepackage{csquotes}
\newcommand{\ts}{\textsuperscript}

% Write initials inside the right margin
\newcommand{\initials}[1]{\marginpar{\quad\texttt{#1}}}

\title{Report of the 3\ts{rd} exercise sheet}
\author{Marco Adamczyk (MA) \and Till Brinkmann (TB)}

% could removed
\date{Submission by 06\ts{th} December 2019}


\begin{document}

\pagenumbering{gobble}

\pagestyle{myheadings}
\markright{Marco Adamczyk, Till Brinkmann}
    
\maketitle

\begin{center}
    \textbf{Tutorial: Tuesday, Tutor: Riza Velioglu}
\end{center}

\section{Introduction}
\subsection{Datasets}
Both datasets have strings of the characters [A..Z] as data, with an average length of 6 characters.
\begin{itemize}
	\item Dataset1 consists of 5493 datapoints. There are 2 different classes. Class 0 appears 4274 times in the set while the other 1219 datapoints are in class 1.
	\item For Dataset2 there are 2002 datapoints in class 0, 2000 in class 1 and 1999 in the third class.
\end{itemize}
\initials{TB}
\subsection{Feature Extraction}
As suggested, we used bigram occurences as our features. This makes the input for the classifiers a vector containing ones or zeros with the size $26^2=676 $.
\initials{TB}

\section{Methods/Models}
\subsection{Naive Bayes Classifier}
	As the first model we chose a Naive Bayes classifier. This model uses a probability given by a simplified Bayes formula (with P(X) always 1) for determining the most likely class for an input X: $$h(x) = \underset{c \in Y}{\arg \max}\ P(X = x | Y = c)*P(Y=c)$$\\
	We defined $P(X = x | Y = c) = 1 - \frac{|x-x_{c,avg}|}{k}$ where $ x,x_{c,avg} \in \mathbb{R}^k, x_{c,avg} = (1\ if\ \frac{1}{|X|}*\sum_{\{x_i \in X | y_i=c\}} x_i > 0.5,\ 0\ otherwise)_k$. This can be described as one minus the normalized distance between x and a median of all $\{x_n \in X| y_n = c\}$.
\initials{TB}

\section{Experiments}
Which tests were done in the experiments? What was implemented? What measurement are used in the results?
\subsection{Data}
Which data are used? What are their characteristics?
\initials{SGM}


\subsection{Results}


\section{Discussion}
Short summary and future work.
\initials{SGM/TGM}

\begin{figure}[h]
\centering
%\includegraphics[width=2.5in]{myfigure}
% where an .eps filename suffix will be assumed under latex, 
% and a .pdf suffix will be assumed for pdflatex; or what has been declared
% via \DeclareGraphicsExtensions.
\caption{Results}
\label{fig_res}
\end{figure}


You can refer to Figure~\ref{fig_res}.
\begin{table}[h]
\caption{An Example of a Table}
\label{tab_example}
\centering
% Some packages, such as MDW tools, offer better commands for making tables
% than the plain LaTeX2e tabular which is used here.
\begin{tabular}{c||c|c|c}
Data & Method 1 & Method 2 & Method 3\\
\hline\hline
data 1 &0.54 & 0.6& 0.98\\
\hline
data 2 &0.74 & 0.54& 0.48\\
\hline
data 3 &0.82 & 0.71& 0.67
\end{tabular}
\end{table}
You can also refer to Table~\ref{tab_example}.
\initials{FGM}

\end{document}